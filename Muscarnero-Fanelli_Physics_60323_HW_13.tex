% Sebastian Muscarnero-Fanelli

\documentclass{article}
\usepackage{graphicx} % Required for inserting images
\usepackage{array} % Required for creating tables
\usepackage{siunitx}
\usepackage{setspace}
\usepackage{amsmath} % for pmatrix
\usepackage{xcolor} % for colored text in math mode

\begin{document}
\begin{center}
\LARGE
PHYS 60323: Fall 2024 - LaTeX Example
\end{center}

\small
\begin{enumerate} \setlength{\leftmargin}{1cm}
    \item \textbf{The following questions refer to stars in the Table below.}
Note: There may be multiple answers.
\begin{center}
\begin{tabular}{ | m{1.2cm} | m{1.2cm} | m{2cm} | m{2.6cm} | m{2cm} | m{1.2cm} |} 
    \hline
    Name & Mass & Luminosity & Lifetime & Temperature & Radius \\ 
    \hline
    \(\eta\) Car. & 60. $M_{\odot}$  &  $ 10^6 L_{\odot}$  & \SI{8.0e5}{years} &   &   \ \\ 
    \hline
    \(\epsilon\) Eri. & 6.0 \(M_\odot\) & $ 10^3 L_{\odot}$ &   & \(20{,}000\) K &   \ \\ 
    \hline
    \(\delta\) Scu. & 2.0 \(M_\odot\) &   & \SI{5.0e8}{years} &   & $ 2 R_{\odot}$ \ \\ 
    \hline
    \(\beta\) Cyg. & 1.3 \(M_\odot\) & $ 3.5 L_{\odot}$ &   &   &   \ \\ 
    \hline
    \(\alpha\) Cen. & 1.0 \(M_\odot\) &   &   &   & $ 1 R_{\odot}$ \ \\ 
    \hline
    \(\gamma\) Del. & 0.7 \(M_\odot\) &   & \SI{4.5e10}{years} & \(5000\) K &   \ \\ 
    \hline
\end{tabular}
\end{center}

\doublespacing

(a) (4 points) Which of these stars will produce a planetary nebula?

(b) (4 points) Elements heavier than $Carbon$ will be produced in which stars?

\item An electron is found to be in the spin state (in the $z$-basis): $\chi = A \begin{pmatrix} 3\mathrm{i} \\ 4 \end{pmatrix}$ %"\usepackage{amsmath} % for pmatrix" must be in the document's code in order for matrices to work.

(a) (5 points) Determine the possible values of $A$ such that the state is normalized.

(b) (5 points) Find the expectation values of the operators $\color{red} S_x$, $\color{purple} S_y$, $\color{orange} S_z$ and $\vec{S}^{\,2}$.

The matrix representations in the $z$-basis for the components of electron spin operators are given by: 

\color{red}\textbf{$S_x$}= $\hbar/2 \begin{pmatrix} 0 & 1 \\ 1 & 0 \end{pmatrix}$ \color{purple}; \hfill \textbf{$S_x$}= $\hbar/2 \begin{pmatrix} 0 & -\mathrm{i} \\ \mathrm{i} & 0 \end{pmatrix}$ \color{orange}; \hfill \textbf{$S_x$}= $\hbar/2 \begin{pmatrix} 1 & 0 \\ 0 & -1 \end{pmatrix}$

\color{black}

\item The average electrostatic field in the earth's atmosphere in fair weather is approximately given: 
\begin{equation}
\vec{E}= E_0(Ae^{-\alpha z}+Be^{-\beta z})\hat{z}
\end{equation}
where A,B,$\alpha$,$\beta$ are positive constants and $z$ is the height above the (locally flat) earth surface.

(a) (5 points) Find the average charge density in the atmosphere as a function of height.

(b) (5 points) Find the electric potential as a function of height above the earth.

\end{enumerate}

\end{document}